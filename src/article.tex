%! Author = kristofferpaulsson
%! Date = 2024-08-03

% Preamble
\documentclass[10pt,a4paper,notitlepage]{article}

\usepackage[margin=2.5cm]{geometry}
\usepackage{fontspec}
\usepackage[none]{hyphenat}  % --- NO HYPHENATION

\usepackage[hyphens]{xurl}  % --- URL HYPHENATION
\usepackage{hyperref}  % --- URL

\usepackage{draftwatermark}  % --- DONT-CITE

\urlstyle{same}  % --- URL
\pagenumbering{gobble}
\DraftwatermarkOptions{fontsize=2em,angle=0,vpos=1cm,color=black,text=Do Not Cite or Circulate}  % --- DONT-CITE

\usepackage{float}  % --- TABLE
\usepackage{makecell}  % --- TABLE
\usepackage{enumitem}  % --- LIST


\fontspec{Tinos}[
    Path = ../../font/t/,
    Extension = .ttf,
    UprightFont = *-Regular,
    Scale=1,
    Ligatures=TeX
]
\fontspec{Inter}[
    Path = ./../font/i/,
    Extension = .ttf ,
    UprightFont = *-Regular ,
    Scale=1,
    Ligatures=TeX
]

% Commands
\newcommand{\tsc}[1]{\textsc{#1}}
\newcommand*{\grc}[1]{\fontspec{Inter}#1\rmfamily}
\newcommand{\trc}[1]{\textit{\fontspec{Tinos}#1}}
\newcommand{\linkfoot}[3]{\footnote{\emph{#1}, s.v. ``{#2},'' accessed \printdate{#3}.}}

% Document
\begin{document}

    \subsection*{Koine Greek Article}
    \begin{itemize}[label={}, leftmargin=0]
        \item Kristoffer E.D.\ Paulsson (2026), Independent Scholar.
    \end{itemize}

    %``\textbf{1099.} The article \grc{ὁ}\fontspec{}, \grc{ἡ}\fontspec{}, \grc{τό}\fontspec{}, was originally a demonstrative pronoun, and as such supplied the place of the personal pronoun of the third person. By gradual weakening it became the definite article\dots''
    %\linkfoot{Article}{https://www.perseus.tufts.edu/hopper/text?doc=Perseus:text:1999.04.0007:part=4:chapter=40}{2025-05-22}

    \begin{table}[H]
        \begin{tabular}{c|ccc}
            \textbf{Inflection} & \textbf{Masc} & \textbf{Fem} & \textbf{Neu} \\
            \hline
            \emph{\tsc{Singular}} \\
            \tsc{Nom} & \makecell{\grc{ὁ} \trc{ho} } & \makecell{\grc{ἡ} \trc{hē} } & \makecell{\grc{τό} \trc{to} } \\
            \tsc{Gen} & \makecell{\grc{τοῦ} \trc{tou} } & \makecell{\grc{τῆς} \trc{tēs} } & \makecell{\grc{τοῦ} \trc{tou} } \\
            \tsc{Dat} & \makecell{\grc{τῷ} \trc{tōi} } & \makecell{\grc{τῇ} \trc{tēi} } & \makecell{\grc{τῷ} \trc{tōi} } \\
            \tsc{Acc} & \makecell{\grc{τόν} \trc{ton} } & \makecell{\grc{τήν} \trc{tēn} } & \makecell{\grc{τό} \trc{to} } \\
            \hline
            \emph{\tsc{Dual}} \\
            \tsc{Nom/Acc} & \makecell{\grc{τώ} \trc{tō} } & \makecell{\grc{τώ} \trc{tō} } & \makecell{\grc{τώ} \trc{tō} } \\
            \tsc{Dat/Gen} & \makecell{\grc{τοῖν} \trc{toin} } & \makecell{\grc{τοῖν} \trc{toin} } & \makecell{\grc{τοῖν} \trc{toin} } \\
            \hline
            \emph{\tsc{Plural}} \\
            \tsc{Nom} & \makecell{\grc{οἱ} \trc{hoi} } & \makecell{\grc{αἱ} \trc{hai} } & \makecell{\grc{τά} \trc{ta} } \\
            \tsc{Gen} & \makecell{\grc{τῶν} \trc{tōn} } & \makecell{\grc{τῶν} \trc{tōn} } & \makecell{\grc{τῶν} \trc{tōn} } \\
            \tsc{Dat} & \makecell{\grc{τοῖς} \trc{tois} } & \makecell{\grc{ταῖς} \trc{tais} } & \makecell{\grc{τοῖς} \trc{tois} } \\
            \tsc{Acc} & \makecell{\grc{τούς} \trc{tous} } & \makecell{\grc{τᾱ́ς} \trc{tas} } & \makecell{\grc{τά} \trc{ta} } \\
        \end{tabular}
        \caption{Compiled from standard Ancient Greek grammars enhanced with transliteration. (Smyth, 1956:94, \S332)}
    \end{table}

    \subsection*{Reference List}
    \begin{itemize}[label={}, leftmargin=0]
       % \item Rydberg-Cox, J. (n.d.). \textit{LESSON XVII: Demonstrative Pronouns}. [online] A Digital Tutorial for Ancient Greek. Available at: \url{https://daedalus.umkc.edu/FirstGreekBook/JWW_FGB17.html} [Accessed 2 Feb. 2026].
       % \item Groton, A.H.\ (2013). \textit{From Alpha to Omega: a Beginning Course in Classical Greek}. Newburyport, Massachusetts: Focus Publishing.
        \item Smyth, H.W.\ (1956). \textit{Greek Grammar}. Cambridge, Massachusetts: Harvard University Press.
    \end{itemize}

\end{document}
