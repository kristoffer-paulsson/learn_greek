%! Author = kristofferpaulsson
%! Date = 2024-08-03

% Preamble
\documentclass[10pt,a4paper,notitlepage]{article}

\usepackage[margin=2.5cm]{geometry}
\usepackage{fontspec}
\usepackage[none]{hyphenat}  % --- NO HYPHENATION

\usepackage[hyphens]{xurl}  % --- URL HYPHENATION
\usepackage{hyperref}  % --- URL

\usepackage{draftwatermark}  % --- DONT-CITE

\urlstyle{same}  % --- URL
\pagenumbering{gobble}
\DraftwatermarkOptions{fontsize=2em,angle=0,vpos=1cm,color=black,text=Do Not Cite or Circulate}  % --- DONT-CITE

\usepackage{float}  % --- TABLE
\usepackage{makecell}  % --- TABLE
\usepackage{enumitem}  % --- LIST


\fontspec{Tinos}[
    Path = ../../font/t/,
    Extension = .ttf,
    UprightFont = *-Regular,
    Scale=1,
    Ligatures=TeX
]
\fontspec{Inter}[
    Path = ./../font/i/,
    Extension = .ttf ,
    UprightFont = *-Regular ,
    Scale=1,
    Ligatures=TeX
]

% Commands
\newcommand{\tsc}[1]{\textsc{#1}}
\newcommand*{\grc}[1]{\fontspec{Inter}#1\rmfamily}
\newcommand{\trc}[1]{\textit{\fontspec{Tinos}#1}}
\newcommand{\linkfoot}[3]{\footnote{\emph{#1}, s.v. ``{#2},'' accessed \printdate{#3}.}}

% Document
\begin{document}

    \subsection*{Koine Greek Medial Demonstrative Pronoun \grc{οὗτος}}
    \begin{itemize}[label={}, leftmargin=0]
        \item Kristoffer E.D.\ Paulsson (2026), Independent Scholar.
    \end{itemize}
    A demonstrative pronoun is a \textit{deictic} element whose reference is determined by the context of utterance,
    reflecting dependence on the speaker’s origo. \textit{Deixis} is commonly divided into spatial, temporal, and
    discursive domains, locating referents relative to the speaker, the time of utterance, or the unfolding text.

    Medial demonstratives place referents at an intermediate distance from the speaker’s deictic center and are
    therefore aligned with a second-person perspective. Ancient Greek has a three-way demonstrative system, in
    which \grc{οὗτος} functions as the \textit{medial} term. It refers to its referents as \textit{that} or,
    more loosely, as \textit{this near you}, and contrasts with the proximal \grc{ὅδε} and the distal \grc{ἐκεῖνος}.

    %Proximal demonstratives place referents near the speaker’s deictic center and are therefore aligned with the
    %first-person perspective. Ancient Greek has a three-way demonstrative system, in which \grc{ὅδε} functions as the
    %strongly \textit{proximal} term. It refers to its referents as \textit{this} or \textit{these} in the sense
    %of \textit{here} and contrasts with the medial \grc{οὗτος} and the distal \grc{ἐκεῖνος}.

    According to Rydberg-Cox (n.d.), it is used ``of something near or present\dots~[or] in referring
    to something which is about to be mentioned.'' Similarly, Groton (2013:77) agrees that \grc{ὅδε}
    ``points out [i.] someone or something very close to the speaker \emph{or} points to [ii.] what
    will follow in the next sentence.''

    \begin{table}[H]
        \begin{tabular}{c|ccc}
            \textbf{Inflection} & \textbf{Masc} & \textbf{Fem} & \textbf{Neu} \\
            \hline
            \emph{\tsc{Singular}} \\
            \tsc{Nom} & \makecell{\grc{οὗτος} \trc{houtos} \\ \small he that} & \makecell{\grc{αὕτη} \trc{hautē} \\ \small she that} & \makecell{\grc{τοῦτο} \trc{touto} \\ \small it that, this one} \\
            \tsc{Gen} & \makecell{\grc{τούτου} \trc{toutoy} \\ \small his that} & \makecell{\grc{ταύτης} \trc{tautēs} \\ \small hers that} & \makecell{\grc{τούτου} \trc{toutoy} \\ \small this one's} \\
            \tsc{Dat} & \makecell{\grc{τούτῳ} \trc{toutōi} \\ \small him that} & \makecell{\grc{ταύτῃ} \trc{tautēi} \\ \small her that} & \makecell{\grc{τούτῳ} \trc{toutōi} \\ \small it that, this one} \\
            \tsc{Acc} & \makecell{\grc{τοῦτον} \trc{touton} \\ \small him that} & \makecell{\grc{ταύτην} \trc{tautēn} \\ \small her that} & \makecell{\grc{τοῦτο} \trc{touto} \\ \small it that, this one} \\
            \hline
            \emph{\tsc{Dual}} \\
            \tsc{Nom/Acc} & \makecell{\grc{τούτω} \trc{toutō} \\ \small these both men} & \makecell{\grc{τούτω} \trc{toutō} \\ \small these both women} & \makecell{\grc{τούτω} \trc{toutō} \\ \small these both} \\
            \tsc{Dat/Gen} & \makecell{\grc{τούτοιν} \trc{toutoin} \\ \small these both men/s} & \makecell{\grc{τούτοιν} \trc{toutoin} \\ \small these both women/s} & \makecell{\grc{τούτοιν} \trc{toutoin} \\ \small these both/theirs} \\
            \hline
            \emph{\tsc{Plural}} \\
            \tsc{Nom} & \makecell{\grc{οὗτοι} \trc{houtoi} \\ \small those men} & \makecell{\grc{αὗται} \trc{hautai} \\ \small those women} & \makecell{\grc{ταῦτα} \trc{tauta} \\ \small those} \\
            \tsc{Gen} & \makecell{\grc{τούτων} \trc{toutōn} \\ \small those men's} & \makecell{\grc{τούτων} \trc{toutōn} \\ \small those women's} & \makecell{\grc{τούτων} \trc{toutōn} \\ \small these's} \\
            \tsc{Dat} & \makecell{\grc{τούτοις} \trc{toutois} \\ \small these men} & \makecell{\grc{ταύταις} \trc{tautais}\\ \small these women} & \makecell{\grc{τούτοις} \trc{toutois} \\ \small these} \\
            \tsc{Acc} & \makecell{\grc{τούτους} \trc{toutoys} \\ \small these men} & \makecell{\grc{ταύτας} \trc{tautas} \\ \small these women} & \makecell{\grc{ταῦτα} \trc{tauta} \\ \small these} \\
        \end{tabular}
        \caption{Compiled from standard Ancient Greek grammars enhanced with transliteration and syntactic cues. (Smyth, 1956:94, §333; Rydberg-Cox, n.d.)}
    \end{table}

    \subsection*{Reference List}
    \begin{itemize}[label={}, leftmargin=0]
        \item Rydberg-Cox, J. (n.d.). \textit{LESSON XVII: Demonstrative Pronouns}. [online] A Digital Tutorial for Ancient Greek. Available at: \url{https://daedalus.umkc.edu/FirstGreekBook/JWW_FGB17.html} [Accessed 2 Feb. 2026].
        \item Groton, A.H.\ (2013). \textit{From Alpha to Omega: a Beginning Course in Classical Greek}. Newburyport, Massachusetts: Focus Publishing.
        \item Smyth, H.W.\ (1956). \textit{Greek Grammar}. Cambridge, Massachusetts: Harvard University Press.
    \end{itemize}

\end{document}
