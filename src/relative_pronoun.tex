%! Author = kristofferpaulsson
%! Date = 2024-08-03

% Preamble
\documentclass[10pt,a4paper,notitlepage]{article}

\usepackage[margin=2.5cm]{geometry}
\usepackage{fontspec}
\usepackage[none]{hyphenat}  % --- NO HYPHENATION

\usepackage[hyphens]{xurl}  % --- URL HYPHENATION
\usepackage{hyperref}  % --- URL

\usepackage{draftwatermark}  % --- DONT-CITE

\urlstyle{same}  % --- URL
\pagenumbering{gobble}
\DraftwatermarkOptions{fontsize=2em,angle=0,vpos=1cm,color=black,text=Do Not Cite or Circulate}  % --- DONT-CITE

\usepackage{float}  % --- TABLE
\usepackage{makecell}  % --- TABLE
\usepackage{enumitem}  % --- LIST


\fontspec{Tinos}[
    Path = ../../font/t/,
    Extension = .ttf,
    UprightFont = *-Regular,
    Scale=1,
    Ligatures=TeX
]
\fontspec{Inter}[
    Path = ./../font/i/,
    Extension = .ttf ,
    UprightFont = *-Regular ,
    Scale=1,
    Ligatures=TeX
]

% Commands
\newcommand{\tsc}[1]{\textsc{#1}}
\newcommand*{\grc}[1]{\fontspec{Inter}#1\rmfamily}
\newcommand{\trc}[1]{\textit{\fontspec{Tinos}#1}}
\newcommand{\linkfoot}[3]{\footnote{\emph{#1}, s.v. ``{#2},'' accessed \printdate{#3}.}}

% Document
\begin{document}

    \subsection*{Ancient Greek: Relative Pronouns}
    \begin{itemize}[label={}, leftmargin=0]
        \item Kristoffer E.D.\ Paulsson (2026), Independent Scholar.
    \end{itemize}

    Rydberg-Cox (n.d.) says that: ``The relative pronouns are \grc{ὅς}, \grc{ἥ}, \grc{ὅ}, \textit{who, which,} and
    \grc{ὅστις}, \grc{ἥτις}, \grc{ὅ τι},
    \textit{whoever, whichever.} The latter is called the indefinite relative.''

    Mounce (2013:90) means that: ``13.22 A \textbf{relative pronoun} is a noun substitute: who(m); whose; that; which;
    what(ever)\dots~The antecedent is the word the pronoun is replacing\dots~13.23 \textbf{Relative clauses} are
    comprised of the relative pronoun, a verb and its subject, and possibly other modifiers. The clause always starts
    with a relative pronoun.''

    \begin{table}[H]
        \begin{tabular}{c|ccc|c}
            & \multicolumn{3}{c|}{Sex (sexus)} & Gender (genus) \\
            \textbf{Inflection} & \textbf{Masc} & \textbf{Fem} & \textbf{Neu} & \textbf{M. F. N.} \\
            \hline
            \emph{\tsc{Singular}} & & & & \\
            \tsc{Nom} & \makecell{\grc{ὅς} \trc{hos} \\ \small he who} & \makecell{\grc{ἥ} \trc{hē} \\ \small she who } & \makecell{\grc{ὅ} \trc{ho} \\ \small who, that} & \makecell{( -- ) \\ \small which/that} \\
            \tsc{Gen} & \makecell{\grc{οὗ} \trc{hou} \\ \small he whose} & \makecell{\grc{ἧς} \trc{hēs} \\ \small she whose} & \makecell{\grc{οὗ} \trc{hou} \\ \small whose} & \makecell{( -- ) \\ \small of which/whose} \\
            \tsc{Dat} & \makecell{\grc{ᾧ} \trc{hōi} \\ \small him to whom } & \makecell{\grc{ᾗ} \trc{hēi} \\ \small her to whom} & \makecell{\grc{ᾧ} \trc{hōi} \\ \small to whom} & \makecell{( -- ) \\ \small to which/whom} \\
            \tsc{Acc} & \makecell{\grc{ὅν} \trc{hon} \\ \small him whom} & \makecell{\grc{ἥν} \trc{hēn} \\ \small her whom} & \makecell{\grc{ὅ} \trc{ho} \\ \small which} & \makecell{( -- ) \\ \small which/that} \\
            \hline
            \emph{\tsc{Dual}} & & & & \\
            \tsc{Nom/Acc} & \makecell{\grc{ὥ} \trc{hō} \\ \small both men who(m)} & \makecell{\grc{ὥ} \trc{hō} \\ \small both women who(m)} & \makecell{\grc{ὥ} \trc{hō} \\ \small both which} & \makecell{( -- ) \\ \small both which/that} \\
            \tsc{Dat/Gen} & \makecell{\grc{οἷν} \trc{hoin} \\ \small both men whose/whom} & \makecell{\grc{οἷν} \trc{hoin} \\ \small both women whose/whom} & \makecell{\grc{οἷν} \trc{hoin} \\ \small both of/to which} & \makecell{( -- ) \\ \small both of which/whose} \\
            \hline
            \emph{\tsc{Plural}} & & & & \\
            \tsc{Nom} & \makecell{\grc{οἵ} \trc{hoi} \\ \small the men who} & \makecell{\grc{αἵ} \trc{hai} \\ \small the women who} & \makecell{\grc{ἅ} \trc{ha} \\ \small those which} & \makecell{( -- ) \\ \small which/that} \\
            \tsc{Gen} & \makecell{\grc{ὧν} \trc{hōn} \\ \small the men whose} & \makecell{\grc{ὧν} \trc{hōn} \\ \small the women whose} & \makecell{\grc{ὧν} \trc{hōn} \\ \small those whose} & \makecell{( -- ) \\ \small of which/whose} \\
            \tsc{Dat} & \makecell{\grc{οἷς} \trc{hois} \\ \small the men to whom} & \makecell{\grc{αἷς} \trc{hais}\\ \small the women to whom} & \makecell{\grc{οἷς} \trc{hois} \\ \small those to which} & \makecell{( -- ) \\ \small to which/whom} \\
            \tsc{Acc} & \makecell{\grc{οὕς} \trc{hous} \\ \small the men whom} & \makecell{\grc{ἅς} \trc{has} \\ \small the women whom} & \makecell{\grc{ἅ} \trc{ha} \\ \small those which} & \makecell{( -- ) \\ \small which/that} \\
        \end{tabular}
        \caption{Compiled from standard Ancient Greek grammars enhanced with transliteration and cues.
            Assume gender over sex except when the antecedent is a person.
            (Smyth, 1956:96--7, \S338)}
    \end{table}

    \subsection*{Reference List}
    \begin{itemize}[label={}, leftmargin=0]
        \item Rydberg-Cox, J. (n.d.). \textit{LESSON LVI: Relative Pronouns. Genitive Absolute. Numerals}. [online] A Digital Tutorial for Ancient Greek. Available at: \url{https://daedalus.umkc.edu/FirstGreekBook/JWW_FGB56.html} [Accessed 6 Feb. 2026].
        \item Mounce, W.D. (2013). \textit{Greek for the Rest of Us: the Essentials of Biblical Greek}. Grand Rapids, Michigan: Zondervan.
        \item Smyth, H.W. (1956). \textit{Greek Grammar}. Cambridge, Massachusetts: Harvard University Press.
    \end{itemize}

\end{document}

