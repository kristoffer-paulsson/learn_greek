%! Author = kristofferpaulsson
%! Date = 2024-08-03

% Preamble
\documentclass[10pt,a4paper,notitlepage]{article}

\usepackage[margin=2.5cm]{geometry}
\usepackage{fontspec}
\usepackage[none]{hyphenat}  % --- NO HYPHENATION

\usepackage[hyphens]{xurl}  % --- URL HYPHENATION
\usepackage{hyperref}  % --- URL

\usepackage{draftwatermark}  % --- DONT-CITE

\urlstyle{same}  % --- URL
\pagenumbering{gobble}
\DraftwatermarkOptions{fontsize=2em,angle=0,vpos=1cm,color=black,text=Do Not Cite or Circulate}  % --- DONT-CITE

\usepackage{float}  % --- TABLE
\usepackage{makecell}  % --- TABLE
\usepackage{enumitem}  % --- LIST


\fontspec{Tinos}[
    Path = ../../font/t/,
    Extension = .ttf,
    UprightFont = *-Regular,
    Scale=1,
    Ligatures=TeX
]
\fontspec{Inter}[
    Path = ./../font/i/,
    Extension = .ttf ,
    UprightFont = *-Regular ,
    Scale=1,
    Ligatures=TeX
]

% Commands
\newcommand{\tsc}[1]{\textsc{#1}}
\newcommand*{\grc}[1]{\fontspec{Inter}#1\rmfamily}
\newcommand{\trc}[1]{\textit{\fontspec{Tinos}#1}}
\newcommand{\linkfoot}[3]{\footnote{\emph{#1}, s.v. ``{#2},'' accessed \printdate{#3}.}}

% Document
\begin{document}

    \subsection*{Koine Greek Distal Demonstrative Pronoun \grc{ἐκεῖνος} for that/those}
    \begin{itemize}[label={}, leftmargin=0]
        \item Kristoffer E.D.\ Paulsson (2026), Independent Scholar.
    \end{itemize}
    A demonstrative pronoun is a \textit{deictic} element whose reference is determined by the context of utterance,
    reflecting dependence on the speaker’s origo. \textit{Deixis} is commonly divided into spatial, temporal, and
    discursive domains, locating referents relative to the speaker, the time of utterance, or the unfolding text.

    Distal demonstratives place referents far from the speaker’s deictic center and are therefore aligned with the
    third-person perspective. Ancient Greek has a three-way demonstrative system, in which \grc{ἐκεῖνος} functions as the
    \textit{distal} term. It refers to its referents as \textit{that} or \textit{those} in the sense of \textit{there}
    and contrasts with the proximal \grc{ὅδε} and the medial \grc{οὗτος}.

    According to Rydberg-Cox (n.d.), ``\grc{ἐκεῖνος}, \textit{that} (yonder), is used of something remote.''
    However, Groton (2013:78) states that it ``points out someone or something far away from the speaker
    \emph{or} labels someone or something as well-known \emph{or} means `the former' (as opposed to `the latter').''

\begin{table}[H]
    \begin{tabular}{c|ccc}
        \textbf{Inflection} & \textbf{Masc} & \textbf{Fem} & \textbf{Neu} \\
        \hline
        \emph{\tsc{Singular}} \\
        \tsc{Nom} & \makecell{\grc{ἐκεῖνος} \trc{ekeinos} \\ \small that (masc.)} & \makecell{\grc{ἐκείνη} \trc{ekeinē} \\ \small that (fem.)} & \makecell{\grc{ἐκεῖνο} \trc{ekeino} \\ \small that (neut.)} \\
        \tsc{Gen} & \makecell{\grc{ἐκείνου} \trc{ekeinou} \\ \small of that (masc.)} & \makecell{\grc{ἐκείνης} \trc{ekeinēs} \\ \small of that (fem.)} & \makecell{\grc{ἐκείνου} \trc{ekeinou} \\ \small of that (neut.)} \\
        \tsc{Dat} & \makecell{\grc{ἐκείνῳ} \trc{ekeinōi} \\ \small to that (masc.)} & \makecell{\grc{ἐκείνῃ} \trc{ekeinēi} \\ \small to that (fem.)} & \makecell{\grc{ἐκείνῳ} \trc{ekeinōi} \\ \small to that (neut.)} \\
        \tsc{Acc} & \makecell{\grc{ἐκεῖνον} \trc{ekeinon} \\ \small that (masc.)} & \makecell{\grc{ἐκείνην} \trc{ekeinēn} \\ \small that (fem.)} & \makecell{\grc{ἐκεῖνο} \trc{ekeino} \\ \small that (neut.)} \\
        \hline
        \emph{\tsc{Dual}} \\
        \tsc{Nom/Acc} & \makecell{\grc{ἐκείνω} \trc{ekeinō} \\ \small those two (masc.)} & \makecell{\grc{ἐκείνω} \trc{ekeinō} \\ \small those two (fem.)} & \makecell{\grc{ἐκείνω} \trc{ekeinō} \\ \small those two (neut.)} \\
        \tsc{Dat/Gen} & \makecell{\grc{ἐκείνοιν} \trc{ekeinoin} \\ \small of/to those two (masc.)} & \makecell{\grc{ἐκείνοιν} \trc{ekeinoin} \\ \small of/to those two (fem.)} & \makecell{\grc{ἐκείνοιν} \trc{ekeinoin} \\ \small of/to those two (neut.)} \\
        \hline
        \emph{\tsc{Plural}} \\
        \tsc{Nom} & \makecell{\grc{ἐκεῖνοι} \trc{ekeinoi} \\ \small those (masc.)} & \makecell{\grc{ἐκεῖναι} \trc{ekeinai} \\ \small those (fem.)} & \makecell{\grc{ἐκεῖνα} \trc{ekeina} \\ \small those (neut.)} \\
        \tsc{Gen} & \makecell{\grc{ἐκείνων} \trc{ekeinōn} \\ \small of those (masc.)} & \makecell{\grc{ἐκείνων} \trc{ekeinōn} \\ \small of those (fem.)} & \makecell{\grc{ἐκείνων} \trc{ekeinōn} \\ \small of those (neut.)} \\
        \tsc{Dat} & \makecell{\grc{ἐκείνοις} \trc{ekeinois} \\ \small to those (masc.)} & \makecell{\grc{ἐκείναις} \trc{ekeinais}\\ \small to those (fem.)} & \makecell{\grc{ἐκείνοις} \trc{ekeinois} \\ \small to those (neut.)} \\
        \tsc{Acc} & \makecell{\grc{ἐκείνους} \trc{ekeinous} \\ \small those (masc.)} & \makecell{\grc{ἐκείνας} \trc{ekeinas} \\ \small those (fem.)} & \makecell{\grc{ἐκεῖνα} \trc{ekeina} \\ \small those (neut.)} \\
    \end{tabular}
    \caption{Compiled from standard Ancient Greek grammars enhanced with transliteration and syntactic cues. (Smyth, 1956:94, \S333; Rydberg-Cox, n.d.)}
\end{table}

    \subsection*{Reference List}
    \begin{itemize}[label={}, leftmargin=0]
        \item Rydberg-Cox, J. (n.d.). \textit{LESSON XVII: Demonstrative Pronouns}. [online] A Digital Tutorial for Ancient Greek. Available at: \url{https://daedalus.umkc.edu/FirstGreekBook/JWW_FGB17.html} [Accessed 2 Feb. 2026].
        \item Groton, A.H.\ (2013). \textit{From Alpha to Omega: a Beginning Course in Classical Greek}. Newburyport, Massachusetts: Focus Publishing.
        \item Smyth, H.W.\ (1956). \textit{Greek Grammar}. Cambridge, Massachusetts: Harvard University Press.
    \end{itemize}

\end{document}