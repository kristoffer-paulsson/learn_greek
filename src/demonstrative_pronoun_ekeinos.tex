%! Author = kristofferpaulsson
%! Date = 2024-08-03

% Preamble
\documentclass[10pt,a4paper,notitlepage]{article}

\usepackage[margin=2.5cm]{geometry}
\usepackage{fontspec}
\usepackage[none]{hyphenat}  % --- NO HYPHENATION

\usepackage[hyphens]{xurl}  % --- URL HYPHENATION
\usepackage{hyperref}  % --- URL

\usepackage{draftwatermark}  % --- DONT-CITE

\urlstyle{same}  % --- URL
\pagenumbering{gobble}
\DraftwatermarkOptions{fontsize=2em,angle=0,vpos=1cm,color=black,text=Do Not Cite or Circulate}  % --- DONT-CITE

\usepackage{float}  % --- TABLE
\usepackage{makecell}  % --- TABLE
\usepackage{enumitem}  % --- LIST


\fontspec{Tinos}[
    Path = ../../font/t/,
    Extension = .ttf,
    UprightFont = *-Regular,
    Scale=1,
    Ligatures=TeX
]
\fontspec{Inter}[
    Path = ./../font/i/,
    Extension = .ttf ,
    UprightFont = *-Regular ,
    Scale=1,
    Ligatures=TeX
]

% Commands
\newcommand{\tsc}[1]{\textsc{#1}}
\newcommand*{\grc}[1]{\fontspec{Inter}#1\rmfamily}
\newcommand{\trc}[1]{\textit{\fontspec{Tinos}#1}}
\newcommand{\linkfoot}[3]{\footnote{\emph{#1}, s.v. ``{#2},'' accessed \printdate{#3}.}}

% Document
\begin{document}

    \subsection*{Koine Greek Distal Demonstrative Pronoun \grc{ἐκεῖνος} for that/those}
    \begin{itemize}[label={}, leftmargin=0]
        \item Kristoffer E.D.\ Paulsson (2026), Independent Scholar.
    \end{itemize}
    A demonstrative pronoun is a \textit{deictic} element whose reference is determined by the context of utterance,
    reflecting dependence on the speaker’s origo. \textit{Deixis} is commonly divided into spatial, temporal, and
    discursive domains, locating referents relative to the speaker, the time of utterance, or the unfolding text.

    Distal demonstratives place referents far from the speaker’s deictic center and are therefore aligned with the
    third-person perspective. Ancient Greek has a three-way demonstrative system, in which \grc{ἐκεῖνος} functions as the
    \textit{distal} term. It refers to its referents as \textit{that} or \textit{those} in the sense of \textit{there}
    and contrasts with the proximal \grc{ὅδε} and the medial \grc{οὗτος}.

    According to Rydberg-Cox (n.d.), ``ἐκεῖνος, \textit{that} (yonder), is used of something remote.''
    However, Groton (2013:78) states that \grc{ἐκεῖνος} ``points out someone or something far away from the speaker
    \emph{or} labels someone or something as well-known \emph{or} means `the former' (as opposed to `the latter').''

    \begin{table}[H]
        \begin{tabular}{c|ccc}
            \textbf{Inflection} & \textbf{Masc} & \textbf{Fem} & \textbf{Neu} \\
            \hline
            \emph{\tsc{Singular}} \\
            \tsc{Nom} & \makecell{\grc{ἐκεῖνος} \trc{ekeinos} \\ \small he that} & \makecell{\grc{ἐκείνη} \trc{ekeinē} \\ \small she that} & \makecell{\grc{ἐκεῖνο} \trc{ekeino} \\ \small it that, this one} \\
            \tsc{Gen} & \makecell{\grc{ἐκείνου} \trc{ekeinou} \\ \small his that} & \makecell{\grc{ἐκείνης} \trc{ekeinēs} \\ \small hers that} & \makecell{\grc{ἐκείνου} \trc{ekeinou} \\ \small this one's} \\
            \tsc{Dat} & \makecell{\grc{ἐκείνῳ} \trc{ekeinōi} \\ \small him that} & \makecell{\grc{ἐκείνῃ} \trc{ekeinēi} \\ \small her that} & \makecell{\grc{ἐκείνῳ} \trc{ekeinōi} \\ \small it that, this one} \\
            \tsc{Acc} & \makecell{\grc{ἐκεῖνον} \trc{ekeinon} \\ \small him that} & \makecell{\grc{ἐκείνην} \trc{ekeinēn} \\ \small her that} & \makecell{\grc{ἐκεῖνο} \trc{ekeino} \\ \small it that, this one} \\
            \hline
            \emph{\tsc{Dual}} \\
            \tsc{Nom/Acc} & \makecell{\grc{ἐκείνω} \trc{ekeinō} \\ \small these both men} & \makecell{\grc{ἐκείνω} \trc{ekeinō} \\ \small these both women} & \makecell{\grc{ἐκείνω} \trc{ekeinō} \\ \small these both} \\
            \tsc{Dat/Gen} & \makecell{\grc{ἐκείνοιν} \trc{ekeinoin} \\ \small these both men/s} & \makecell{\grc{ἐκείνοιν} \trc{ekeinoin} \\ \small these both women/s} & \makecell{\grc{ἐκείνοιν} \trc{ekeinoin} \\ \small these both/theirs} \\
            \hline
            \emph{\tsc{Plural}} \\
            \tsc{Nom} & \makecell{\grc{ἐκεῖνοι} \trc{ekeinoi} \\ \small those men} & \makecell{\grc{ἐκεῖναι} \trc{ekeinai} \\ \small those women} & \makecell{\grc{ἐκεῖνα} \trc{ekeina} \\ \small those} \\
            \tsc{Gen} & \makecell{\grc{ἐκείνων} \trc{ekeinōn} \\ \small those men's} & \makecell{\grc{ἐκείνων} \trc{ekeinōn} \\ \small those women's} & \makecell{\grc{ἐκείνων} \trc{ekeinōn} \\ \small these's} \\
            \tsc{Dat} & \makecell{\grc{ἐκείνοις} \trc{ekeinois} \\ \small these men} & \makecell{\grc{ἐκείναις} \trc{ekeinais}\\ \small these women} & \makecell{\grc{ἐκείνοις} \trc{ekeinois} \\ \small these} \\
            \tsc{Acc} & \makecell{\grc{ἐκείνους} \trc{ekeinous} \\ \small these men} & \makecell{\grc{ἐκείνας} \trc{ekeinas} \\ \small these women} & \makecell{\grc{ἐκεῖνα} \trc{ekeina} \\ \small these} \\
        \end{tabular}
        \caption{Compiled from standard Ancient Greek grammars enhanced with transliteration and syntactic cues. (Smyth, 1956:94, §333; Rydberg-Cox, n.d.)}
    \end{table}

    \subsection*{Reference List}
    \begin{itemize}[label={}, leftmargin=0]
        \item Rydberg-Cox, J. (n.d.). \textit{LESSON XVII: Demonstrative Pronouns}. [online] A Digital Tutorial for Ancient Greek. Available at: \url{https://daedalus.umkc.edu/FirstGreekBook/JWW_FGB17.html} [Accessed 2 Feb. 2026].
        \item Groton, A.H.\ (2013). \textit{From Alpha to Omega: a Beginning Course in Classical Greek}. Newburyport, Massachusetts: Focus Publishing.
        \item Smyth, H.W.\ (1956). \textit{Greek Grammar}. Cambridge, Massachusetts: Harvard University Press.
    \end{itemize}

\end{document}