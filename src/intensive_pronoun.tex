%! Author = kristofferpaulsson
%! Date = 2024-08-03

% Preamble
\documentclass[10pt,a4paper,notitlepage]{article}

\usepackage[margin=2.5cm]{geometry}
\usepackage{fontspec}
\usepackage[none]{hyphenat}  % --- NO HYPHENATION

\usepackage[hyphens]{xurl}  % --- URL HYPHENATION
\usepackage{hyperref}  % --- URL

\usepackage{draftwatermark}  % --- DONT-CITE

\urlstyle{same}  % --- URL
\pagenumbering{gobble}
\DraftwatermarkOptions{fontsize=2em,angle=0,vpos=1cm,color=black,text=Do Not Cite or Circulate}  % --- DONT-CITE

\usepackage{float}  % --- TABLE
\usepackage{makecell}  % --- TABLE
\usepackage{enumitem}  % --- LIST


\fontspec{Tinos}[
    Path = ../../font/t/,
    Extension = .ttf,
    UprightFont = *-Regular,
    Scale=1,
    Ligatures=TeX
]
\fontspec{Inter}[
    Path = ./../font/i/,
    Extension = .ttf ,
    UprightFont = *-Regular ,
    Scale=1,
    Ligatures=TeX
]

% Commands
\newcommand{\tsc}[1]{\textsc{#1}}
\newcommand*{\grc}[1]{\fontspec{Inter}#1\rmfamily}
\newcommand{\trc}[1]{\textit{\fontspec{Tinos}#1}}
\newcommand{\linkfoot}[3]{\footnote{\emph{#1}, s.v. ``{#2},'' accessed \printdate{#3}.}}

% Document
\begin{document}
    \subsection*{Ancient Greek: Intensive Pronouns}
    \begin{itemize}[label={}, leftmargin=0]
        \item Kristoffer E.D.\ Paulsson (2026), Independent Scholar.
    \end{itemize}


    \begin{quote}
    % Exact citation start
    \grc{αὐτός} is a definite adjective and a pronoun. It has three meanings:
    \textbf{a.} \textit{self:} standing by itself in the nominative, \grc{αὐτὸς} \grc{ὁ} \grc{ἀνήρ} or
    \grc{ὁ} \grc{ἀνὴρ} \grc{αὐτός} the man himself, or (without the article) in agreement with a substantive or
    pronoun; as \grc{ἀνδρὸς} \grc{αὐτοῦ} of the man himself.
    \textbf{b.} \textit{him, her, it, them}, etc.: standing by itself in an oblique case (never in the nominative).
    The oblique cases of \grc{αὐτός} are generally used instead of \grc{οὗ}, \grc{οἷ}, \grc{ἕ}, etc., as
    \grc{ὁ} \grc{πατὴρ} \grc{αὐτοῦ} his father, \grc{οἱ} \grc{παῖδες} \grc{αὐτῶν} their children.
    \textbf{c.} \textit{same:} when it is preceded by the article in any case: \grc{ὁ} \grc{αὐτὸς} \grc{ἀνήρ}
    the same man, \grc{τοῦ} \grc{αὐτοῦ} of the same man.
    % Exact citation stop
    (Smyth, 1956:92, \S328)
    \end{quote}

    Example usage of \grc{αὐταῖν} in genitive case with natural gender: \emph{both women's}; or with grammatical gender: \emph{of both things}.

    \begin{table}[H]
        \begin{tabular}{c|ccc}
            \textbf{Inflection} & \textbf{Masc} & \textbf{Fem} & \textbf{Neu} \\
            \hline
            \emph{\tsc{Singular}} \\
            \tsc{Nom} & \makecell{\grc{αὐτός} \trc{autos} \\ \small himself} & \makecell{\grc{αὐτή} \trc{autē} \\ \small herself} & \makecell{\grc{αὐτό} \trc{auto} \\ \small itself} \\
            \tsc{Gen} & \makecell{\grc{αὐτοῦ} \trc{autou} \\ \small his own} & \makecell{\grc{αὐτῆς} \trc{autēs} \\ \small her own} & \makecell{\grc{αὐτοῦ} \trc{autou} \\ \small its own} \\
            \tsc{Dat} & \makecell{\grc{αὐτῷ} \trc{autōi} \\ \small to/for himself} & \makecell{\grc{αὐτῇ} \trc{autēi} \\ \small to/for herself} & \makecell{\grc{αὐτῷ} \trc{autōi} \\ \small to/for itself} \\
            \tsc{Acc} & \makecell{\grc{αὐτόν} \trc{auton} \\ \small himself} & \makecell{\grc{αὐτήν} \trc{autēn} \\ \small herself} & \makecell{\grc{αὐτό} \trc{auto} \\ \small itself} \\
            \hline
            \emph{\tsc{Dual}} \\
            \tsc{Nom/Acc} & \makecell{\grc{αὐτώ} \trc{autō} \\ \small both (m.)} & \makecell{\grc{αὐτά} \trc{auta} \\ \small both (f.)} & \makecell{\grc{αὐτώ} \trc{autō} \\ \small both (n.)} \\
            \tsc{Dat/Gen} & \makecell{\grc{αὐτοῖν} \trc{autoin} \\ \small of/to/for both (m.)} & \makecell{\grc{αὐταῖν} \trc{autain} \\ \small of/to/for both (f.)} & \makecell{\grc{αὐτοῖν} \trc{autoin} \\ \small of/to/for both (n.)} \\
            \hline
            \emph{\tsc{Plural}} \\
            \tsc{Nom} & \makecell{\grc{αὐτοί} \trc{autoi} \\ \small they (m.)} & \makecell{\grc{αὐταί} \trc{autai} \\ \small they (f.)} & \makecell{\grc{αὐτά} \trc{auta} \\ \small they (n.)} \\
            \tsc{Gen} & \makecell{\grc{αὐτῶν} \trc{autōn} \\ \small their (m.)} & \makecell{\grc{αὐτῶν} \trc{autōn} \\ \small their (f.)} & \makecell{\grc{αὐτῶν} \trc{autōn} \\ \small their (n.)} \\
            \tsc{Dat} & \makecell{\grc{αὐτοῖς} \trc{autois} \\ \small to/for them (m.)} & \makecell{\grc{αὐταῖς} \trc{autais}\\ \small to/for them (f.)} & \makecell{\grc{αὐτοῖς} \trc{autois} \\ \small to/for them (n.)} \\
            \tsc{Acc} & \makecell{\grc{αὐτούς} \trc{autous} \\ \small them (m.)} & \makecell{\grc{αὐτάς} \trc{autas} \\ \small them (f.)} & \makecell{\grc{αὐτά} \trc{auta} \\ \small them (n.)} \\
        \end{tabular}
         \caption{Compiled from standard Ancient Greek grammars enhanced with transliteration and syntactic cues. (Smyth, 1956:92, \S327)}
     \end{table}

    \subsection*{Reference List}
    \begin{itemize}[label={}, leftmargin=0]
        \item Rydberg-Cox, J. (n.d.). \textit{LESSON XLVII: Personal Pronouns}. [online] A Digital Tutorial for Ancient Greek. Available at: \url{https://daedalus.umkc.edu/FirstGreekBook/JWW_FGB47.html} [Accessed 5 Feb. 2026].
        \item Cambridge Dictionary (2024). \textit{Yous}. [online] @CambridgeWords. Available at: \url{https://dictionary.cambridge.org/dictionary/english/yous} [Accessed 13 May 2025].
        \item Smyth, H.W.\ (1956). \textit{Greek Grammar}. Cambridge, Massachusetts: Harvard University Press.
    \end{itemize}

\end{document}
