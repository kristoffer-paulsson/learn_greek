%! Author = kristofferpaulsson
%! Date = 2024-08-03

% Preamble
\documentclass[10pt,a4paper,notitlepage]{article}

\usepackage[margin=2.5cm]{geometry}
\usepackage{fontspec}
\usepackage[none]{hyphenat}  % --- NO HYPHENATION

\usepackage[hyphens]{xurl}  % --- URL HYPHENATION
\usepackage{hyperref}  % --- URL

\usepackage{draftwatermark}  % --- DONT-CITE

\urlstyle{same}  % --- URL
\pagenumbering{gobble}
\DraftwatermarkOptions{fontsize=2em,angle=0,vpos=1cm,color=black,text=Do Not Cite or Circulate}  % --- DONT-CITE

\usepackage{float}  % --- TABLE
\usepackage{makecell}  % --- TABLE
\usepackage{enumitem}  % --- LIST


\fontspec{Tinos}[
    Path = ../../font/t/,
    Extension = .ttf,
    UprightFont = *-Regular,
    Scale=1,
    Ligatures=TeX
]
\fontspec{Inter}[
    Path = ./../font/i/,
    Extension = .ttf ,
    UprightFont = *-Regular ,
    Scale=1,
    Ligatures=TeX
]

% Commands
\newcommand{\tsc}[1]{\textsc{#1}}
\newcommand*{\grc}[1]{\fontspec{Inter}#1\rmfamily}
\newcommand{\trc}[1]{\textit{\fontspec{Tinos}#1}}
\newcommand{\linkfoot}[3]{\footnote{\emph{#1}, s.v. ``{#2},'' accessed \printdate{#3}.}}

% Document
\begin{document}
    \subsection*{Koine Greek Personal Pronouns First and Second Person}
    \begin{itemize}[label={}, leftmargin=0]
        \item Kristoffer E.D.\ Paulsson (2026), Independent Scholar.
    \end{itemize}


    ``First-person pronouns are words such as `I' and `us' that refer either to the person who said or wrote them (singular), or to a group including the speaker or writer (plural).''
    \linkfoot{First-Person Pronouns}{https://www.scribbr.com/academic-writing/first-person-pronouns/}{2025-05-12}

     ``Second-person pronouns are words like `you' that refer to the person or people being spoken or written to.''
    \linkfoot{Second-Person Pronouns}{https://www.scribbr.com/academic-writing/second-person-pronouns/}{2025-05-13}

    To differentiate between singular and plural ``you'' in English, the modern plural ``yous''\linkfoot{yous}{https://dictionary.cambridge.org/dictionary/english/yous}{2025-05-13}
    and ``yourselves''\linkfoot{yourselves}{https://dictionary.cambridge.org/dictionary/english/yourselves}{2025-05-13}
    has been adpoted.

     \begin{table}[H]
         \begin{tabular}{c|cc}
             \textbf{Inflection} & \textbf{First} & \textbf{Second} \\
             \hline
             \emph{\tsc{Singular}} \\
             \tsc{Nom} & \makecell{\grc{ἐγώ} \trc{egō} \\ \small I} & \makecell{\grc{σύ} \trc{sy} \\ \small you} \\
             \tsc{Gen} & \makecell{\grc{ἐμοῦ} \trc{emou}, \grc{μου} \trc{mou} \\ \small my, mine} & \makecell{\grc{σοῦ} \trc{sou}, \grc{σου} \trc{sou} \\ \small your, yours} \\
             \tsc{Dat} & \makecell{\grc{ἐμοί} \trc{emoi}, \grc{μοι} \trc{moi} \\ \small to\/for me} & \makecell{\grc{σοί} \trc{soi}, \grc{σοι} \trc{soi} \\ \small to\/for you} \\
             \tsc{Acc} & \makecell{\grc{ἐμέ} \trc{eme}, \grc{με} \trc{me} \\ \small me} & \makecell{\grc{σέ} \trc{se}, \grc{σε} \trc{se} \\ \small you} \\
             \hline
             \emph{\tsc{Dual}} \\
             \tsc{Nom/Acc} & \makecell{\grc{νώ} \trc{nō} \\ \small we two; both of us} & \makecell{\grc{σφώ} \trc{sphō} \\ \small you two; both of you} \\
             \tsc{Dat/Gen} & \makecell{\grc{νῷν} \trc{nōin} \\ \small of\/to both of us} & \makecell{\grc{σφῷν} \trc{sphōin} \\ \small of\/to both of you} \\
             \hline
             \emph{\tsc{Plural}} \\
             \tsc{Nom} & \makecell{\grc{ἡμεῖς} \trc{hēmeis} \\ \small we} & \makecell{\grc{ὑμεῖς} \trc{hymeis} \\ \small yous} \\
             \tsc{Gen} & \makecell{\grc{ἡμῶν} \trc{hēmōn} \\ \small our, ours} & \makecell{\grc{ὑμῶν} \trc{hymōn} \\ \small your, yours} \\
             \tsc{Dat} & \makecell{\grc{ἡμῖν} \trc{hēmin} \\ \small to\/for us} & \makecell{\grc{ὑμῖν} \trc{hymin} \\ \small to\/for yous} \\
             \tsc{Acc} & \makecell{\grc{ἡμᾶς} \trc{hēmas} \\ \small us} & \makecell{\grc{ὑμᾶς} \trc{hymas} \\ \small yous} \\
         \end{tabular}
         \caption{Compiled from standard Ancient Greek grammars enhanced with transliteration and syntactic cues. (Smyth, 1956:90, \S325)}
     \end{table}

    \subsection*{Reference List}
    \begin{itemize}[label={}, leftmargin=0]
       % \item Rydberg-Cox, J. (n.d.). \textit{LESSON XVII: Demonstrative Pronouns}. [online] A Digital Tutorial for Ancient Greek. Available at: \url{https://daedalus.umkc.edu/FirstGreekBook/JWW_FGB17.html} [Accessed 2 Feb. 2026].
       % \item Groton, A.H.\ (2013). \textit{From Alpha to Omega: a Beginning Course in Classical Greek}. Newburyport, Massachusetts: Focus Publishing.
        \item Smyth, H.W.\ (1956). \textit{Greek Grammar}. Cambridge, Massachusetts: Harvard University Press.
    \end{itemize}

\end{document}
