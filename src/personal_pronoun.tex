%! Author = kristofferpaulsson
%! Date = 2024-08-03

% Preamble
\documentclass[10pt,a4paper,notitlepage]{article}

\usepackage[margin=2.5cm]{geometry}
\usepackage{fontspec}
\usepackage[none]{hyphenat}  % --- NO HYPHENATION

\usepackage[hyphens]{xurl}  % --- URL HYPHENATION
\usepackage{hyperref}  % --- URL

\usepackage{draftwatermark}  % --- DONT-CITE

\urlstyle{same}  % --- URL
\pagenumbering{gobble}
\DraftwatermarkOptions{fontsize=2em,angle=0,vpos=1cm,color=black,text=Do Not Cite or Circulate}  % --- DONT-CITE

\usepackage{float}  % --- TABLE
\usepackage{makecell}  % --- TABLE
\usepackage{enumitem}  % --- LIST


\fontspec{Tinos}[
    Path = ../../font/t/,
    Extension = .ttf,
    UprightFont = *-Regular,
    Scale=1,
    Ligatures=TeX
]
\fontspec{Inter}[
    Path = ./../font/i/,
    Extension = .ttf ,
    UprightFont = *-Regular ,
    Scale=1,
    Ligatures=TeX
]

% Commands
\newcommand{\tsc}[1]{\textsc{#1}}
\newcommand*{\grc}[1]{\fontspec{Inter}#1\rmfamily}
\newcommand{\trc}[1]{\textit{\fontspec{Tinos}#1}}
\newcommand{\linkfoot}[3]{\footnote{\emph{#1}, s.v. ``{#2},'' accessed \printdate{#3}.}}

% Document
\begin{document}
    \subsection*{Ancient Greek: Personal Pronouns in First and Second Person}
    \begin{itemize}[label={}, leftmargin=0]
        \item Kristoffer E.D.\ Paulsson (2026), Independent Scholar.
    \end{itemize}
    Personal pronouns are parts of speech used to refer to the speaker, the addressee, or others in a conversation.
    First-person pronouns indicate the speaker, while second-person pronouns indicate the addressee. They decline
    for \textit{case} and \textit{number}, while \textit{person} is inherent and \textit{gender} is generally absent
    except in later third-person forms.
    Because finite verbs mark person and number, nominative pronouns are often omitted, appearing mainly for emphasis,
    contrast, or clarification. Oblique cases are regularly used to mark objects or possession. Pronouns also show a
    distinction between strong (emphatic) and weak (enclitic) forms, reflecting accentuation and discourse
    prominence rather than grammatical necessity.

    Rydberg-Cox (n.d.) says ``the nominatives of the personal pronouns are seldom used, except for emphasis.'' Furthermore,
    ``the forms \grc{μοῦ}, \grc{μοί}, \grc{μέ}; \grc{σοῦ}, \grc{σοί}, \grc{σέ}\dots~are enclitic.
    But if the pronoun is emphatic, the enclitic forms of the pronoun retain their accent, and in the first person
    the longer forms \grc{ἐμοῦ}, \grc{ἐμοί}, \grc{ἐμέ}, are then used. This generally happens also after prepositions.''

    To differentiate between singular and plural \emph{you} in English, the modern plural \emph{yous} has been
    adopted (Cambridge Dictionary, 2024).

    % and ``yourselves''\linkfoot{yourselves}{https://dictionary.cambridge.org/dictionary/english/yourselves}{2025-05-13}

     \begin{table}[H]
         \begin{tabular}{c|ccc}
             \textbf{Inflection} & \textbf{1st} & \textbf{2nd} & \textbf{3rd} \\
             \hline
             \emph{\tsc{Singular}} \\
             \tsc{Nom} & \makecell{\grc{ἐγώ} \trc{egō} \\ \small I} & \makecell{\grc{σύ} \trc{sy} \\ \small you} \\
             \tsc{Gen} & \makecell{\grc{ἐμοῦ} \trc{emou}, \grc{μου} \trc{mou} \\ \small my, mine} & \makecell{\grc{σοῦ} \trc{sou}, \grc{σου} \trc{sou} \\ \small your, yours}  & \makecell{\grc{οὗ} \trc{hou}, \grc{οὑ} \trc{hou} \\ \small ( -- )} \\
             \tsc{Dat} & \makecell{\grc{ἐμοί} \trc{emoi}, \grc{μοι} \trc{moi} \\ \small to/for me} & \makecell{\grc{σοί} \trc{soi}, \grc{σοι} \trc{soi} \\ \small to/for you}  & \makecell{\grc{οἷ} \trc{hoi}, \grc{οἱ} \trc{hoi} \\ \small ( -- )} \\
             \tsc{Acc} & \makecell{\grc{ἐμέ} \trc{eme}, \grc{με} \trc{me} \\ \small me} & \makecell{\grc{σέ} \trc{se}, \grc{σε} \trc{se} \\ \small you}  & \makecell{\grc{ἕ} \trc{he}, \grc{ἑ} \trc{he} \\ \small ( -- )} \\
             \hline
             \emph{\tsc{Dual}} \\
             \tsc{Nom/Acc} & \makecell{\grc{νώ} \trc{nō} \\ \small we two; both of us} & \makecell{\grc{σφώ} \trc{sphō} \\ \small you two; both of you} \\
             \tsc{Dat/Gen} & \makecell{\grc{νῷν} \trc{nōin} \\ \small of/to/for us both} & \makecell{\grc{σφῷν} \trc{sphōin} \\ \small of/to/for you both} \\
             \hline
             \emph{\tsc{Plural}} \\
             \tsc{Nom} & \makecell{\grc{ἡμεῖς} \trc{hēmeis} \\ \small we} & \makecell{\grc{ὑμεῖς} \trc{humeis} \\ \small yous}  & \makecell{\grc{--} \trc{--} \\ \small ( -- )} \\
             \tsc{Gen} & \makecell{\grc{ἡμῶν} \trc{hēmōn} \\ \small our, ours} & \makecell{\grc{ὑμῶν} \trc{humōn} \\ \small your, yours}  & \makecell{\grc{--} \trc{--} \\ \small ( -- )} \\
             \tsc{Dat} & \makecell{\grc{ἡμῖν} \trc{hēmin} \\ \small to/for us} & \makecell{\grc{ὑμῖν} \trc{humin} \\ \small to/for yous}  & \makecell{\grc{--} \trc{--} \\ \small ( -- )} \\
             \tsc{Acc} & \makecell{\grc{ἡμᾶς} \trc{hēmas} \\ \small us} & \makecell{\grc{ὑμᾶς} \trc{humas} \\ \small yous}  & \makecell{\grc{--} \trc{--} \\ \small ( -- )} \\
         \end{tabular}
         \caption{Compiled from standard Ancient Greek grammars enhanced with transliteration and syntactic cues. (Smyth, 1956:90, \S325)}
     \end{table}

    \subsection*{Reference List}
    \begin{itemize}[label={}, leftmargin=0]
        \item Rydberg-Cox, J. (n.d.). \textit{LESSON XLVII: Personal Pronouns}. [online] A Digital Tutorial for Ancient Greek. Available at: \url{https://daedalus.umkc.edu/FirstGreekBook/JWW_FGB47.html} [Accessed 5 Feb. 2026].
        \item Cambridge Dictionary (2024). \textit{Yous}. [online] @CambridgeWords. Available at: \url{https://dictionary.cambridge.org/dictionary/english/yous} [Accessed 13 May 2025].
        \item Smyth, H.W.\ (1956). \textit{Greek Grammar}. Cambridge, Massachusetts: Harvard University Press.
    \end{itemize}

\end{document}
