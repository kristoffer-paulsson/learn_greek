%! Author = kristofferpaulsson
%! Date = 2024-08-03

% Preamble
\documentclass[10pt,a4paper,notitlepage]{article}

\usepackage[margin=2.5cm]{geometry}
\usepackage{fontspec}
\usepackage[none]{hyphenat}  % --- NO HYPHENATION

\usepackage[hyphens]{xurl}  % --- URL HYPHENATION
\usepackage{hyperref}  % --- URL

\usepackage{draftwatermark}  % --- DONT-CITE

\urlstyle{same}  % --- URL
\pagenumbering{gobble}
\DraftwatermarkOptions{fontsize=2em,angle=0,vpos=1cm,color=black,text=Do Not Cite or Circulate}  % --- DONT-CITE

\usepackage{float}  % --- TABLE
\usepackage{makecell}  % --- TABLE
\usepackage{enumitem}  % --- LIST


\fontspec{Tinos}[
    Path = ../../font/t/,
    Extension = .ttf,
    UprightFont = *-Regular,
    Scale=1,
    Ligatures=TeX
]
\fontspec{Inter}[
    Path = ./../font/i/,
    Extension = .ttf ,
    UprightFont = *-Regular ,
    Scale=1,
    Ligatures=TeX
]

% Commands
\newcommand{\tsc}[1]{\textsc{#1}}
\newcommand*{\grc}[1]{\fontspec{Inter}#1\rmfamily}
\newcommand{\trc}[1]{\textit{\fontspec{Tinos}#1}}
\newcommand{\linkfoot}[3]{\footnote{\emph{#1}, s.v. ``{#2},'' accessed \printdate{#3}.}}

% Document
\begin{document}
    \subsection*{Ancient Greek: Interrogative Pronouns}
    \begin{itemize}[label={}, leftmargin=0]
        \item Kristoffer E.D.\ Paulsson (2026), Independent Scholar.
    \end{itemize}

    ``\textbf{334.} Interrogative and Indefinite Pronouns. --The interrogative pronoun \grc{τίς}\fontspec{}, \grc{τί} \fontspec{} who, which, what? \dots''
    \linkfoot{Interrogative, Smyth §334}{https://www.perseus.tufts.edu/hopper/text?doc=Perseus\%3Atext\%3A1999.04.0007\%3Apart\%3D2\%3Achapter\%3D15}{2025-05-24}

    ``Interrogative pronouns are pronouns that are used to ask questions. The main English interrogative pronouns are what, which, who, whom, and whose.''
        \linkfoot{Interrogative Pronouns}{https://www.scribbr.com/nouns-and-pronouns/interrogative-pronouns/}{2025-05-24}

    \begin{table}[H]
        \begin{tabular}{c|cc}
            \textbf{Inflection} & \textbf{Masc/Fem} & \textbf{Neu} \\
            \hline
            \emph{\tsc{Singular}} \\
            \tsc{Nom} & \makecell{\grc{τίς} \trc{tis} \\ \small one who} & \makecell{\grc{τί} \trc{ti} \\ \small that which } \\
            \tsc{Gen} & \makecell{\grc{τίνος} \trc{tinos} \\ \small one whose} & \makecell{\grc{τίνος} \trc{tinos} \\ \small which} \\
            \tsc{Dat} & \makecell{\grc{τίνι} \trc{tini} \\ \small one whom } & \makecell{\grc{τίνι} \trc{tini} \\ \small what} \\
            \tsc{Acc} & \makecell{\grc{τίνα} \trc{tina} \\ \small one whom} & \makecell{\grc{τίνα} \trc{tina} \\ \small what} \\
            \hline
            \emph{\tsc{Dual}} \\
            \tsc{Nom/Acc} & \makecell{\grc{τίνε} \trc{tine} \\ \small both who} & \makecell{\grc{τινέ} \trc{tine} \\ \small both which} \\
            \tsc{Dat/Gen} & \makecell{\grc{τίνοιν} \trc{tinoin} \\ \small both who(se)} & \makecell{\grc{τινοῖν} \trc{tinoin} \\ \small both which} \\
            \hline
            \emph{\tsc{Plural}} \\
            \tsc{Nom} & \makecell{\grc{τίνες} \trc{tines} \\ \small those who} & \makecell{\grc{τίνα} \trc{tina} \\ \small these which} \\
            \tsc{Gen} & \makecell{\grc{τίνων} \trc{tinōn} \\ \small those whose} & \makecell{\grc{τίνων} \trc{tinōn} \\ \small these whose} \\
            \tsc{Dat} & \makecell{\grc{τίσι(ν)} \trc{tisi(n)} \\ \small they whom} & \makecell{\grc{τίσι(ν)} \trc{tisi(n)}\\ \small these which} \\
            \tsc{Acc} & \makecell{\grc{τίνας} \trc{tinas} \\ \small they whom} & \makecell{\grc{τίνα} \trc{tina} \\ \small these which} \\
        \end{tabular}
         \caption{Compiled from standard Ancient Greek grammars enhanced with transliteration and syntactic cues. (Smyth, 1956:96, \S334)}
     \end{table}

    \subsection*{Reference List}
    \begin{itemize}[label={}, leftmargin=0]
        %\item Rydberg-Cox, J. (n.d.). \textit{LESSON XLVII: Personal Pronouns}. [online] A Digital Tutorial for Ancient Greek. Available at: \url{https://daedalus.umkc.edu/FirstGreekBook/JWW_FGB47.html} [Accessed 5 Feb. 2026].
        %\item Cambridge Dictionary (2024). \textit{Yous}. [online] @CambridgeWords. Available at: \url{https://dictionary.cambridge.org/dictionary/english/yous} [Accessed 13 May 2025].
        \item Smyth, H.W.\ (1956). \textit{Greek Grammar}. Cambridge, Massachusetts: Harvard University Press.
    \end{itemize}

\end{document}
