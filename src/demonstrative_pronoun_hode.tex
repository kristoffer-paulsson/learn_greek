%! Author = kristofferpaulsson
%! Date = 2024-08-03

% Preamble
\documentclass[10pt,a4paper,notitlepage]{article}

\usepackage[margin=2.5cm]{geometry}
\usepackage{fontspec}
\usepackage[none]{hyphenat}  % --- NO HYPHENATION

\usepackage[hyphens]{xurl}  % --- URL HYPHENATION
\usepackage{hyperref}  % --- URL

\usepackage{draftwatermark}  % --- DONT-CITE

\urlstyle{same}  % --- URL
\pagenumbering{gobble}
\DraftwatermarkOptions{fontsize=2em,angle=0,vpos=1cm,color=black,text=Do Not Cite or Circulate}  % --- DONT-CITE

\usepackage{float}  % --- TABLE
\usepackage{makecell}  % --- TABLE
\usepackage{enumitem}  % --- LIST


\fontspec{Tinos}[
    Path = ../../font/t/,
    Extension = .ttf,
    UprightFont = *-Regular,
    Scale=1,
    Ligatures=TeX
]
\fontspec{Inter}[
    Path = ./../font/i/,
    Extension = .ttf ,
    UprightFont = *-Regular ,
    Scale=1,
    Ligatures=TeX
]

% Commands
\newcommand{\tsc}[1]{\textsc{#1}}
\newcommand*{\grc}[1]{\fontspec{Inter}#1\rmfamily}
\newcommand{\trc}[1]{\textit{\fontspec{Tinos}#1}}
\newcommand{\linkfoot}[3]{\footnote{\emph{#1}, s.v. ``{#2},'' accessed \printdate{#3}.}}

% Document
\begin{document}

    \subsection*{Koine Greek Proximal Demonstrative Pronoun \grc{ὅδε} for this/these here}
    \begin{itemize}[label={}, leftmargin=0]
        \item Kristoffer E.D.\ Paulsson (2026), Independent Scholar.
    \end{itemize}
    A demonstrative pronoun is a \textit{deictic} element whose reference is determined by the context of utterance,
    reflecting dependence on the speaker’s origo. \textit{Deixis} is commonly divided into spatial, temporal, and
    discursive domains, locating referents relative to the speaker, the time of utterance, or the unfolding text.

    Proximal demonstratives place referents near the speaker’s deictic center and are therefore aligned with the
    first-person perspective. Ancient Greek has a three-way demonstrative system, in which \grc{ὅδε} functions as the
    strongly \textit{proximal} term. It refers to its referents as \textit{this} or \textit{these} in the sense
    of \textit{here} and contrasts with the medial \grc{οὗτος} and the distal \grc{ἐκεῖνος}.

    According to Rydberg-Cox (n.d.), it is used ``of something near or present\dots~[or] in referring
    to something which is about to be mentioned.'' Similarly, Groton (2013:77) agrees that \grc{ὅδε}
    ``points out [i.] someone or something very close to the speaker \emph{or} points to [ii.] what
    will follow in the next sentence.''

    Example usage of \grc{τοῖνδε} in genitive case with natural gender: \emph{of both these women here}; or with grammatical gender: \emph{here of these two things}.

    \begin{table}[H]
        \begin{tabular}{c|ccc}
            \textbf{Inflection} & \textbf{Masc} & \textbf{Fem} & \textbf{Neu} \\
            \hline
            \emph{\tsc{Singular}} \\
            \tsc{Nom} & \makecell{\grc{ὅδε} \trc{hode} \\ \small this (here)} & \makecell{\grc{ἥδε} \trc{hēde} \\ \small this (here)} & \makecell{\grc{τόδε} \trc{tode} \\ \small this (here)} \\
            \tsc{Gen} & \makecell{\grc{τοῦδε} \trc{toude} \\ \small of this (here)} & \makecell{\grc{τῆσδε} \trc{tēsde} \\ \small of this (here)} & \makecell{\grc{τοῦδε} \trc{toude} \\ \small of this (here)} \\
            \tsc{Dat} & \makecell{\grc{τῷδε} \trc{tōide} \\ \small to/for this (here)} & \makecell{\grc{τῇδε} \trc{tēide} \\ \small to/for this (here)} & \makecell{\grc{τῷδε} \trc{tōide} \\ \small to/for this (here)} \\
            \tsc{Acc} & \makecell{\grc{τόνδε} \trc{tonde} \\ \small this (here)} & \makecell{\grc{τήνδε} \trc{tēnde} \\ \small this (here)} & \makecell{\grc{τόδε} \trc{tode} \\ \small this (here)} \\
            \hline
            \emph{\tsc{Dual}} \\
            \tsc{Nom/Acc} & \makecell{\grc{τώδε} \trc{tōde} \\ \small these two (here)} & \makecell{\grc{τώδε} \trc{tōde} \\ \small these two (here)} & \makecell{\grc{τώδε} \trc{tōde} \\ \small these two (here)} \\
            \tsc{Dat/Gen} & \makecell{\grc{τοῖνδε} \trc{toinde} \\ \small of/to/for these two (here)} & \makecell{\grc{τοῖνδε} \trc{toinde} \\ \small of/to/for these two (here)} & \makecell{\grc{τοῖνδε} \trc{toinde} \\ \small of/to/for these two (here)} \\
            \hline
            \emph{\tsc{Plural}} \\
            \tsc{Nom} & \makecell{\grc{οἵδε} \trc{hoide} \\ \small these (here)} & \makecell{\grc{αἵδε} \trc{haide} \\ \small these (here)} & \makecell{\grc{τάδε} \trc{tade} \\ \small these (here)} \\
            \tsc{Gen} & \makecell{\grc{τῶνδε} \trc{tōnde} \\ \small of these (here)} & \makecell{\grc{τῶνδε} \trc{tōnde} \\ \small of these (here)} & \makecell{\grc{τῶνδε} \trc{tōnde} \\ \small of these (here)} \\
            \tsc{Dat} & \makecell{\grc{τοῖσδε} \trc{toisde} \\ \small to/for these (here)} & \makecell{\grc{ταῖσδε} \trc{taisde}\\ \small to/for these (here)} & \makecell{\grc{τοῖσδε} \trc{toisde} \\ \small to/for these (here)} \\
            \tsc{Acc} & \makecell{\grc{τούσδε} \trc{tousde} \\ \small these (here)} & \makecell{\grc{τάσδε} \trc{tasde} \\ \small these (here)} & \makecell{\grc{τάδε} \trc{tade} \\ \small these (here)} \\
        \end{tabular}
        \caption{Compiled from standard Ancient Greek grammars enhanced with transliteration and syntactic cues. (Smyth, 1956:94, \S333; Rydberg-Cox, n.d.)}
    \end{table}

    \subsection*{Reference List}
    \begin{itemize}[label={}, leftmargin=0]
        \item Rydberg-Cox, J. (n.d.). \textit{LESSON XVII: Demonstrative Pronouns}. [online] A Digital Tutorial for Ancient Greek. Available at: \url{https://daedalus.umkc.edu/FirstGreekBook/JWW_FGB17.html} [Accessed 2 Feb. 2026].
        \item Groton, A.H.\ (2013). \textit{From Alpha to Omega: a Beginning Course in Classical Greek}. Newburyport, Massachusetts: Focus Publishing.
        \item Smyth, H.W.\ (1956). \textit{Greek Grammar}. Cambridge, Massachusetts: Harvard University Press.
    \end{itemize}

\end{document}
