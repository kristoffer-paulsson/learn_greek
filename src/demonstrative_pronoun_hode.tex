%! Author = kristofferpaulsson
%! Date = 2024-08-03

% Preamble
\documentclass[11pt,a4paper,notitlepage]{article}

\usepackage[margin=2.5cm]{geometry}
\usepackage{fontspec}
\usepackage[none]{hyphenat}  % --- NO HYPHENATION

\usepackage[hyphens]{xurl}  % --- URL HYPHENATION
\usepackage{hyperref}  % --- URL

\usepackage{draftwatermark}  % --- DONT-CITE

\urlstyle{same}  % --- URL
\pagenumbering{gobble}
\DraftwatermarkOptions{fontsize=2em,angle=0,vpos=1cm,color=black,text=Do Not Cite or Circulate}  % --- DONT-CITE

\usepackage{float}  % --- TABLE
\usepackage{makecell}  % --- TABLE
\usepackage{enumitem}  % --- LIST


\fontspec{Tinos}[
    Path = ../../font/t/,
    Extension = .ttf,
    UprightFont = *-Regular,
    Scale=1,
    Ligatures=TeX
]
\fontspec{Inter}[
    Path = ./../font/i/,
    Extension = .ttf ,
    UprightFont = *-Regular ,
    Scale=1,
    Ligatures=TeX
]

% Commands
\newcommand{\tsc}[1]{\textsc{#1}}
\newcommand*{\grc}[1]{\fontspec{Inter}#1\rmfamily}
\newcommand{\trc}[1]{\textit{\fontspec{Tinos}#1}}
\newcommand{\linkfoot}[3]{\footnote{\emph{#1}, s.v. ``{#2},'' accessed \printdate{#3}.}}

% Document
\begin{document}

    \section*{Koine Greek Proximal Demonstrative Pronoun \grc{ὅδε}
    }

    A \textit{demonstrative pronoun} is a deictic element whose reference is determined by the context of utterance.
    \textit{Deixis} captures this dependence on the speaker’s \textit{origo} and is commonly divided into spatial,
    temporal, and discursive domains. A deixis may locate referents relative to the speaker, to the time of
    utterance, or within the unfolding text.
    \textit{Proximal demonstratives} places proximity to the speaker’s deictic center, and thus aligned with
    the first-person perspective.
    Ancient Greek has a three-way demonstrative system. \grc{ὅδε} is the strongly
    proximal term, which relates to its referents in terms of \emph{this} or \emph{these} in the sense of \emph{here},
    and contrasts with the medial \grc{οὗτος}, and the distal \grc{ἐκεῖνος}.

    According to Rydberg-Cox (n.d.) it is used ``of something near or present\dots [or] in referring
    to something which is about to be mentioned,'' also Groton (2013:77) agrees to that \grc{ὅδε}
    ``points out [i.] someone or something very close to the speaker \emph{or} points to [ii.] what
    will follow in the next sentence.''

    \begin{table}[H]
        \begin{tabular}{c|ccc}
            \textbf{Inflection} & \textbf{Masc} & \textbf{Fem} & \textbf{Neu} \\
            \hline
            \emph{\tsc{Singular}} \\
            \tsc{Nom} & \makecell{\grc{ὅδε} \trc{hode} \\ \small he that} & \makecell{\grc{ἥδε} \trc{hēde} \\ \small she that} & \makecell{\grc{τόδε} \trc{tode} \\ \small it that, this one} \\
            \tsc{Gen} & \makecell{\grc{τοῦδε} \trc{toude} \\ \small his that} & \makecell{\grc{τῆσδε} \trc{tēsde} \\ \small hers that} & \makecell{\grc{τοῦδε} \trc{toude} \\ \small this one's} \\
            \tsc{Dat} & \makecell{\grc{τῷδε} \trc{tōide} \\ \small him that} & \makecell{\grc{τῇδε} \trc{tēide} \\ \small her that} & \makecell{\grc{τῷδε} \trc{tōide} \\ \small it that, this one} \\
            \tsc{Acc} & \makecell{\grc{τόνδε} \trc{tonde} \\ \small him that} & \makecell{\grc{τήνδε} \trc{tēnde} \\ \small her that} & \makecell{\grc{τόδε} \trc{tode} \\ \small it that, this one} \\
            \hline
            \emph{\tsc{Dual}} \\
            \tsc{Nom/Acc} & \makecell{\grc{τώδε} \trc{tōde} \\ \small these both men} & \makecell{\grc{τώδε} \trc{tōde} \\ \small these both women} & \makecell{\grc{τώδε} \trc{tōde} \\ \small these both} \\
            \tsc{Dat/Gen} & \makecell{\grc{τοῖνδε} \trc{toinde} \\ \small these both men/s} & \makecell{\grc{τοῖνδε} \trc{toinde} \\ \small these both women/s} & \makecell{\grc{τοῖνδε} \trc{toinde} \\ \small these both/theirs} \\
            \hline
            \emph{\tsc{Plural}} \\
            \tsc{Nom} & \makecell{\grc{οἵδε} \trc{hoide} \\ \small those men} & \makecell{\grc{αἵδε} \trc{haide} \\ \small those women} & \makecell{\grc{τάδε} \trc{tade} \\ \small those} \\
            \tsc{Gen} & \makecell{\grc{τῶνδε} \trc{tōnde} \\ \small those men's} & \makecell{\grc{τῶνδε} \trc{tōnde} \\ \small those women's} & \makecell{\grc{τῶνδε} \trc{tōnde} \\ \small these's} \\
            \tsc{Dat} & \makecell{\grc{τοῖσδε} \trc{toisde} \\ \small these men} & \makecell{\grc{ταῖσδε} \trc{taisde}\\ \small these women} & \makecell{\grc{τοῖσδε} \trc{toisde} \\ \small these} \\
            \tsc{Acc} & \makecell{\grc{τούσδε} \trc{tousde} \\ \small these men} & \makecell{\grc{τάσδε} \trc{tasde} \\ \small these women} & \makecell{\grc{τάδε} \trc{tade} \\ \small these} \\
        \end{tabular}
        \caption{Compiled from standard Ancient Greek grammars enhanced with transliteration and syntactic cues. (Smyth, 1956:94, §333; Rydberg-Cox, n.d.)}
    \end{table}

    \section*{Reference List}

    \begin{itemize}[label={}, leftmargin=0]
        \item Rydberg-Cox, J. (n.d.). \textit{LESSON XVII: Demonstrative Pronouns}. [online] A Digital Tutorial for Ancient Greek. Available at: \url{https://daedalus.umkc.edu/FirstGreekBook/JWW_FGB17.html} [Accessed 2 Feb. 2026].
        \item Groton, A.H.\ (2013). \textit{From Alpha to Omega: a Beginning Course in Classical Greek}. Newburyport, Massachusetts: Focus Publishing.
        \item Smyth, H.W.\ (1956). \textit{Greek Grammar}. Cambridge, Massachusetts: Harvard University Press.
    \end{itemize}

\end{document}
