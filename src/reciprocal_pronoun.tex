%! Author = kristofferpaulsson
%! Date = 2024-08-03

% Preamble
\documentclass[10pt]{memoir}

% Packages
\usepackage{amsmath}
\usepackage{float}
% Please add the following required packages to your document preamble:
\usepackage{multirow}
\usepackage[T1]{fontenc}
\usepackage{times}
\usepackage{fontspec}
\usepackage{makecell}

\fontspec{Tinos}[
    Path = ../../font/t/,
    Extension = .ttf,
    UprightFont = *-Regular,
    Scale=1,
    Ligatures=TeX
]
\fontspec{Inter}[
    Path = ./../font/i/,
    Extension = .ttf ,
    UprightFont = *-Regular ,
    Scale=1,
    Ligatures=TeX
]

%\setromanfont{Tinos}

\newcommand{\tsc}[1]{\textsc{#1}}
\newcommand{\grc}[1]{\fontspec{Inter}#1}
\newcommand{\trc}[1]{\textit{\fontspec{Tinos}#1}}
\newcommand{\linkfoot}[3]{\footnote{\emph{#1}, s.v. ``{#2},'' accessed \printdate{#3}.}}

\renewcommand{\arraystretch}{1.3}

\usepackage{draftwatermark}
\usepackage{hyperref}
\usepackage{hyperref}
\SetWatermarkFontSize{1.3em}
\DraftwatermarkOptions{angle=0,vpos=1.5cm,color=black,text=Do Not Cite or Circulate}

% Document
\begin{document}

    \section*{Koine Greek Reciprocal Pronouns}

    ``\textbf{331.} Reciprocal Pronoun. --The reciprocal pronoun, meaning one another, each other, is made
    by doubling the stem of \grc{ἄλλος} (\grc{ἀλλαλλο}). \fontspec{} It is used only in the oblique cases of
    the dual and plural. (Cp. alii aliorum, alter alterius).''
    \linkfoot{Reciprocal Pronouns, Smyth §331}{https://www.perseus.tufts.edu/hopper/text?doc=Perseus\%3Atext\%3A1999.04.0007\%3Apart\%3D2\%3Achapter\%3D15}{2025-05-23}

    \begin{table}[H]
        \begin{tabular}{c|ccc}
            \textbf{Inflection} & \textbf{Masc} & \textbf{Fem} & \textbf{Neu} \\
            \hline
            \emph{\tsc{Dual}} \\
            \tsc{Acc} & \makecell{\grc{ἀλλήλω} \trc{allēlō} \\ \small both each men} & \makecell{\grc{ἀλλήλα} \trc{allēla} \\ \small both each women} & \makecell{\grc{ἀλλήλω} \trc{allēlō} \\ \small both each of them} \\
            \tsc{Dat/Gen} & \makecell{\grc{ἀλλήλοιν} \trc{allēloin} \\ \small both each men/s} & \makecell{\grc{ἀλλήλαιν} \trc{allēlain} \\ \small both each women/s} & \makecell{\grc{ἀλλήλοιν} \trc{allēloin} \\ \small both each of them/theirs} \\
            \hline
            \emph{\tsc{Plural}} \\
            \tsc{Gen} & \makecell{\grc{ἀλλήλων} \trc{allēlōn} \\ \small each men's} & \makecell{\grc{ἀλλήλων} \trc{allēlōn} \\ \small each women's} & \makecell{\grc{ἀλλήλων} \trc{allēlōn} \\ \small each of theirs} \\
            \tsc{Dat} & \makecell{\grc{ἀλλήλοις} \trc{allēlois} \\ \small each other men} & \makecell{\grc{ἀλλήλαις} \trc{allēlais}\\ \small each other women} & \makecell{\grc{ἀλλήλοις} \trc{allēlois} \\ \small each other of them } \\
            \tsc{Acc} & \makecell{\grc{ἀλλήλους} \trc{allēlous} \\ \small each other men} & \makecell{\grc{ἀλλήλας} \trc{allēlas} \\ \small each other women} & \makecell{\grc{ἄλληλα} \trc{allēla} \\ \small each others} \\
        \end{tabular}
        \caption{Reciprocal pronouns according similar to Smyth §331, p. 94.}
    \end{table}
\end{document}
